\section{Related Work}
\label{sec:related_work}
In the following sections, a literature review is conducted. The first two parts describe the current scientific approaches to measure emotions for both faces and voices using deep neural networks. The third part explains the relationship between the quality of presentations and emotions.

\subsection{Facial Emotion Recognition}
\label{subsec:related_work_facial_emotion_recognition}
Facial Emotion Recognition (FER) is a subarea of computer vision and deals with the prediction of human emotional states using facial mimics and expressions in images, moving pictures or videos \cite{jain_extended_2019}.
According to \citeA{rosler_reducing_2021} FER literature and research suggest firstly, two separate ways of feature generation: manually or automatically through a deep neural network. Secondly, they outline the differentiation of their underlying emotional model, which is either based on discrete emotional states or on continuous dimensions \cite{rosler_reducing_2021}. Five fundamental discrete emotional states, for example identified by \citeA{ekman_universal_1997}, are happiness, sadness, fear / surprise, disgust and anger. There is several different literature recognizing other fundamental emotional states, and thus a set definition is not available. Also, the continuous dimensions are defined divergently. Given an instance, \citeA{mollahosseini_affectnet_2017} use two or three dimensions to explain emotions, with valence or pleasantness as one dimension and arousal or activation as the other.

In this work we will focus on leveraging discrete emotional states, since it is more maturely researched and more suitable to generate features using deep learning based approaches. Considering the amount of developments of recent years in deep learning, we concentrate on using the available variety of deep neural networks as the most appropriate technique in FER \cite{jain_extended_2019}. We could also focus on FER approaches that use handcrafted features and which are according to \citeA{ko_brief_2018} often deployed in the following three steps. First, face and facial component detection from the input image. Second, spatial and temporal feature extraction. And third, expression classification by e.g. Support Vector Machine or Random Forests to recognize emotional states. Although this manual way often leads to accurate results and requires less computational resources, many researches, such as \citeA{jung_joint_2015}, showed the superiority of deep learning algorithms with for example CNNs over handcrafted approaches.

One of the main advantages of CNNs and other neural networks is the possibility to automatically learn features from the input images, which is called ``end-to-end'' learning \cite{ko_brief_2018}. This Moody Prototype also uses a CNN implementation via the \texttt{face-api.js} library, which is described further in Section~\ref{subsec:method_facial_emotion_recognition}. These in the context of facial emotion recognition most used CNNs operate and process images, as the name discloses, ``convolutionally'' and can take spatial information into consideration \cite{ko_brief_2018, rosler_reducing_2021}. Thus, we utilize a CNN to recognize the emotional states of \citeA{ekman_universal_1997}: happy, sad, fearful, angry, surprised and disgusted. Additionally, since audiences listen and their faces appear often simply neutrally, we added another emotional state to our recognition model, called ``neutral''.

\subsection{Vocal Emotion Recognition}
\label{subsec:related_work_vocal_emotion_recognition}
The use of emotional impulses that portray emotion in a single modality, typically through facial expressions, has become popular in emotion research. Emotional communication in the natural world, on the other hand, is temporal and multimodal. Multisensory integration is important while processing effective cues, according to research \cite{livingstone_ryerson_2018}. Researchers have created their own multimodal stimuli in the absence of proven multimodal sets (Livingstone and Russo 2018). Researchers have also combined two different unimodal sets \cite{delle-vigne_subclinical_2014} or joined self-created stimuli with an existing unimodal set \cite{zvyagintsev_attention_2013} to create multimodal stimuli. Because each set differs in features, technical quality, and expressive intensity, comparing findings across studies may be difficult. As a result, differences in results could be attributed in part to differences in stimulus sets \cite{livingstone_ryerson_2018}.

In their research paper, \citeA{livingstone_ryerson_2018} described the creation and validation of the RAVDESS, a set of dynamic and multimodal emotional emotions. The RAVDESS dataset has a number of key features that make it ideal for scientists, engineers, and physicians to use: It is provided freely available under a Creative Commons non-commercial license and features professional performers from North America. It has a variety of emotional reactions at two levels of emotional intensity \cite{livingstone_ryerson_2018}. 

The RAVDESS was validated with 247 raters from across North America. The accuracy with which participants properly identified the actors' intended emotions was referred to as validity. As is customary in the literature, \citeA{livingstone_ryerson_2018} looked at proportion correct scores. Overall, the results were excellent, with an average of 80\% for audio-video, 75\% for video-only, and 60\% for audio-only (Livingstone and Russo 2018).

Next to the RAVDESS Dataset, there exist other voice datasets to train CNN for voice emotion Recognition. The JL-Corpus is a set of four New Zealand English speakers who each say 15 sentences in each of five basic emotions with two repetitions, and another 10 sentences in each of five secondary emotions (An Open Source Emotional Speech Corpus for Human Robot Interaction Applications).

Another example is the Toronto Emotion Speech Set (TESS). It includes sentences of 200 words,  which always have the same structure \cite{pichora-fuller_toronto_2020}. The carrier phrase ``Say the word ...'' was spoken and recorded by two actors (aged 26 and 64), and these were each divided into seven different emotions (anger, disgust, fear, happiness, pleasant surprise, sadness and neutral) \cite{pichora-fuller_toronto_2020}. Together, this resulted in 2800 different stimuli. The two actors were from the Toronto area and their mother language was English  \cite{pichora-fuller_toronto_2020}. The EMO-DB database is a free emotional database. The Institute of Communication Science, Technical University of Berlin, Germany, established the database. The data sets consist of  the voices of ten german professional speakers (five males and five females), which display seven different emotions(anger, boredom,  anxiety, happiness, sadness, disgust, and neutral). There are 800 sentences recorded and seven emotions in the EMO-DB database \cite{burkhardt_database_2005}.

Because there are already so many valid data sets, we will use the mentioned four for our application and train our CNN on it. 

\subsection{Presentations and Emotions}
\label{subsec:related_work_presentations_and_emotions}
As stated by \citeA{derrico_tracking_2019} great speeches or presentations depend significantly on if the presenter transmits enough emotions while transferring their information. When compared to emotionless presentations, presentations that leverage emotional experiences and elicit an emotional response from the audience are more likely to capture the audience's attention (Gallo, 2014). Further facts prove that emotions are an important driver for engaging presentations and are a crucial enabler for successful knowledge transfer. Brain researchers have discovered that presented knowledge will be remembered more likely if the presenter communicates emotionally (Tyng et al., 2017). Moreover, politicians frequently postulate wrath and despair in order to convey empathy and concern about the matter at hand \cite{derrico_tracking_2019}. Also, most people do not judge brands with facts and data, but rely heavily on feelings and emotions they have about the brand (Damasio et al., 2006).

An exemplary study by \citeA{de_carolis_engaged_2019} researched a tool for emotion recognition from facial expressions during e-learnings and comes close to our developed tool concerning the facial emotion recognition part. By evaluating facial expressions, head movements, and gaze behavior from 5,5-hour video recordings, the authors were able to automatically quantify students’ engagement. The information gathered was linked to a subjective evaluation of engagement based on a four-dimensional questionnaire: challenge, skill, engagement, and perceived learning. According to Carolis et al. (2019), the less anxious and the more relaxed students are, the more they evaluated a more engaged questionnaire. They also showed that the more excited and engaged students felt during a presentation, the more the emotional analysis perceived them like this.

Another study of Chen et al. (2014) analyzed besides nonverbal behaviors also the speech delivery, such as fluency, pronunciation, and prosody. Here the authors collected several presentations from different speakers, which then were evaluated by human raters in different dimensions concerning the felt engagement during the presentation. They found out that indeed machine learning algorithms on vocal analysis can be used to assess the performance of presentations. Nevertheless this study did not explicitly take into account emotion recognition, but shows that the way the presenter delivers his speech correlates with the effectiveness of presentations.

Our paper project differs from previous studies, since there has not yet been a tool developed, which can recognize emotions from two input sources: facial and vocal expressions. With this tool at hand, and previous findings in literature that emotions are an important driver for successful presentations, future presenters could leverage their information conveyment through exactly knowing the audiences’ current emotional state.
