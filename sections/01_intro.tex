\section{Introduction}
\label{sec:intro}
The global pandemic caused by the Covid-19 virus had a strong impact on the professional environment in 2020 and in the first half of 2021. Employers sent their staff into the home office as far as possible. A survey found that before the Corona crisis, around 40 percent of employees in companies worked from home in the home office. During the pandemic, this share increased by about 20 percentage points to around 60 percent. This has sometimes meant that normal business meetings, whether with clients or colleagues, have had to take place via video conferencing platforms such as ZOOM or similar. Nevertheless, in the business world, as well as in university life, presentations are an essential part of sharing knowledge. It is important to present in the right way to have the best possible impact on the audience. For a good presentation, the way in which the presenters express their information using emotions is therefore of essential importance \cite{derrico_tracking_2019}.

Emotions in presentations can be transmitted in various ways, such as vocabulary and emotional words like “grateful”, “unique”, and “honored” when expressing inspiration, whether the presenter communicates enthusiastically vs. indifferently, or the provided gestures and facial expressions while presenting, such as smiling, being surprised or sad etc \cite{zeng_emoco_2019, rosler_reducing_2021}. Using deep neural networks, in particular Convolutional Neural Networks (CNNs), it is possible to determine emotions from facial expressions with a high degree of accuracy. Many researchers have shown, for example, that their deep learning-based methods perform very well on various datasets for recognising emotions in the face \cite{ko_brief_2018} (Ko 2018). Emotion regarding voice can also be determined using such neural networks. Using facial emotion recognition, some researchers analysed the relationship between facial expressions and the learning process to monitor and measure student engagement (De Carolis, 2019), for example, to provide personalised feedback to improve the learning experience. 

In our seminar paper, we look more closely at the relationship between emotion through facial expressions and through voice in presentation. With the help of CNNs and trained emotion recognition models, we built a web app (\url{www.moody.digital}) that measures and stores both facial and vocal emotions in real-time during live presentations. In addition, feedback from the participants is requested after each presentation in order to put the presentations into context. Such an evaluation could help researchers and practitioners to better understand the inherent relationship between emotions and presentations, and furthermore to improve the quality of presentations in the long term, thereby promoting better knowledge transfer. Therefore, our research question is as follows:
\vspace{3mm}
\newline
\emph{(RQ)\quad How can we design a system to collect real-time feedback with the goal to reduce video conference fatigue?}
