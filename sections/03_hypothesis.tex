\section{Hypothesis}
\label{sec:hypothesis}
Our hypothesis is based on the findings of Rößler et al (2021), who used CNN networks and face emotion recognition to find that the presenter must always maintain a positive attitude to convey enthusiasm and positive energy to the audience. This was studied using only face motion recognition technology. Our hypothesis goes further in this regard, and we are therefore also investigating the emotional tone of voice of the presenter and its influence on the feedback from the participants of the presentation. With the support of both FER and vocal emotion recognition, and the subsequent feedback from the participants, we then want to build an open-source application, which is showing the presenter live the facial and voice emotions of the audience. Therefore, the presenter has a tool to improve his presentation live or also later looking at the feedback and the collected emotions. The developed tool then can also operate as a data collection tool, to get further information about online virtual meetings, and about what makes it for the audience a good or a bad presentation. Thereby, even lengthy video presentations will lead to positive experiences for the audience and thus a similar experience for the presenter.
