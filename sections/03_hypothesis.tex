\section{Hypothesis}
\label{sec:hypothesis}
Our hypothesis is based on the findings of \citeA{rosler_reducing_2021}, who used CNNs and face emotion recognition to find that the presenter must always maintain a positive attitude to convey enthusiasm and positive energy to the audience. This was studied using only face emotion recognition technology. Our hypothesis goes further in this regard, and we are therefore also investigating the emotional tone of voice of the presenter and its influence on the feedback from the participants of the presentation. With the support of both FER and vocal emotion recognition, and the subsequent feedback from the participants, we want to build an open source application which is showing the presenter facial and voice emotions of the audience in real-time. Therefore, the presenter has a tool to improve his presentation live or also later looking at the feedback and the collected emotions. The application can also operate as a data collection tool, to get further information about online virtual meetings, and about what makes it for the audience a good or a bad presentation. Thereby, users of our application are given the possibility to improve the experience even during lengthy video presentations.
